%-------------------------------------
% Resume in LaTeX (XeLaTeX)
% Author:  Anthony Meza
% Base template by: HF Yuan
% Project: https://github.com/xyz-yuanhf/yuan-resume
% License: MIT
%------------------------------------

\documentclass[a4paper, 11pt]{article}
\usepackage{simpleresume}  % Style package

% --------------------  START  --------------------
\begin{document}

% -------------------- HEADING --------------------
\headerContact{(714) 552-2396}{ameza98@outlook.com}{anthony-meza.github.io}
\headerName{Anthony Meza}
\headerTitle{Ph.D. Candidate}

% -------------------- RESEARCH INTERESTS --------------------
% \section{Research Interests}

% \skillListStart
% \justifying
% \hyphenpenalty=10000
% \exhyphenpenalty=10000
% \item My thesis is focused on past and future changes in global ocean circulation, with an emphasis on identifying the driving mechanisms and evaluating their detectability in observational records. Beyond my thesis work, I am broadly interested in all research that combine models and observations to advance our understanding of Earth's global and regional climate systems, particularly problems involving ocean dynamics, climate variability, and long-term climate change.\\
% \skillListEnd

% -------------------- EDUCATION --------------------
\section{Education}

\eduHeading
  {\parbox[t]{0.7\textwidth}{Massachusetts Institute of Technology}}{Cambridge, MA}
  {Ph.D. in Physical Oceanography}{Expected 2026}
\eduHeading
  {University of California, Irvine}{Irvine, CA}
  {B.S. in Mathematics - Concentration in Data Science}{2018--2021}
\eduHeading
  {Fullerton College}{Fullerton, CA}
  {A.S. in Mathematics}{2016--2018}

% -------------------- RESEARCH EXPERIENCE --------------------
\section{Experience}

\positionHeading[Graduate Research Assistant]
  {Woods Hole Oceanographic Institution}{Woods Hole, MA}{Sep 2021--Present}
\itemListStart
  \myItem{Designed and ran global ocean simulations using the MIT General Circulation Model (MITgcm) to test mechanisms controlling deep ocean heat content and circulation}
\myItem{Evaluated high-resolution coupled climate models to quantify the impacts of Antarctic sea ice melt on global ocean circulation and tracer distributions}
\myItem{Analyzed ocean reanalysis data and found a statistical relationship between near-shore sea surface temperature variability and extreme California precipitation events}
\myItem{Developed Python and Julia tools for processing and analysis of ocean model and observational data in high-performance computing (HPC) environments}
\itemListEnd

\positionHeading[Technical Consultant]
  {Foundation for Resilient Societies}{Cambridge, MA}{Jan 2025}
\itemListStart
  % \myItem{Received in-person training from Astrapé Consulting on the Strategic Energy \& Risk Valuation Model (SERVM) for electric grid capacity adequacy analysis}
\myItem{Ran and debugged Strategic Energy \& Risk Valuation Model (SERVM) simulations to assess U.S. electrical grid capacity adequacy under varying generation scenarios (e.g., solar adoption).}
\myItem{Led a team of 12 undergraduate electric grid modeling interns to develop an internal user guide for running SERVM experiments and interpreting model output.}
\itemListEnd

\positionHeading[Research Intern]
  {Los Alamos National Laboratory}{Los Alamos, NM}{Jun 2021--Aug 2021}
\itemListStart
  \myItem{Implemented and evaluated reduced-precision in the Energy Exascale Earth System Model (E3SM) to reduce computational cost and energy consumption in global climate simulations}
\itemListEnd

\positionHeading[Research Intern]
  {\parbox[t]{0.7\textwidth}{Institute for Pure and Applied Mathematics {\&}\\The Aerospace Corporation}}{Los Angeles, CA}{Jun 2020--Sep 2020}
\itemListStart
    \myItem{Designed and implemented reinforcement learning–based methods for adaptive packet routing in satellite network simulations, implemented in Python using PyTorch}
\itemListEnd

% -------------------- PUBLICATIONS --------------------
\section{Publications}

\pubListStart
\justifying
\pubItem{\textbf{Meza, A.}, \& Gebbie, G. (2025). Wind-driven mid-depth Pacific cooling in a dynamically consistent ocean state estimate. \textit{Journal of Geophysical Research: Oceans}. doi.org/10.1029/2025JC022462}
% \item \textbf{Meza, A.}, Drake, H. \& Gebbie, G. \textit{(In Prep)}. Projected Changes in Dense Shelf Water and their Abyssal Expression.
% \item \textbf{Meza, A.}, et al. \textit{(In Prep)}. 
% Detection of .
% \item \textbf{Meza, A.}, Bhuyan, P., Zheng, Z. \& Gebbie, G. \textit{(In Prep)}. Passive and Dynamic Contributions to Mean Ocean Temperature during the Last Glacial Maximum.
\pubListEnd

% -------------------- PROJECTS --------------------
\section{Personal Projects}
\ProjectHeading{xbuoy}
\itemListStart
\myItem{Developed \textit{xbuoy}, a Python workflow to query National Data Buoy Center (NDBC) and aggregate irregularly sampled data into commonly used Earth science data formats (e.g., NetCDF).}
\itemListEnd


% -------------------- SKILLS --------------------
\section{Skills}
\skillListStart
\justifying
\item \emph{Languages}: \textit{Programming}: Python, Julia, MATLAB; \textit{Human}: English, Spanish
\item \emph{Scientific Computing}: NumPy, SciPy, xarray, Pandas,  Optimization.jl, JuMP.jl, scikit-learn, PyTorch
\item \emph{HPC \& Dev Tools}: Unix/Linux, OpenMPI, HPC job schedulers (e.g., Slurm), Dask, Git, GitHub
\skillListEnd


\end{document}
