%-------------------------------------
% Resume in LaTeX (XeLaTeX)
% Author:  Anthony Meza
% Base template by: HF Yuan
% Project: https://github.com/xyz-yuanhf/yuan-resume
% License: MIT
%------------------------------------

\documentclass[a4paper, 11pt]{article}
\usepackage{simpleresume}  % Style package

% --------------------  START  --------------------
\begin{document}

% -------------------- HEADING --------------------
\headerContact{(714) 552-2396}{ameza@mit.edu}{anthony-meza.github.io}
\headerName{Anthony Meza}
\headerTitle{Ph.D. Candidate}

% -------------------- RESEARCH INTERESTS --------------------
% \section{Research Interests}

% \skillListStart
% \justifying
% \hyphenpenalty=10000
% \exhyphenpenalty=10000
% \item My thesis is focused on past and future changes in global ocean circulation, with an emphasis on identifying the driving mechanisms and evaluating their detectability in observational records. Beyond my thesis work, I am broadly interested in all research that combine models and observations to advance our understanding of Earth's global and regional climate systems, particularly problems involving ocean dynamics, climate variability, and long-term climate change.\\
% \skillListEnd

% -------------------- EDUCATION --------------------
\section{Education}

\eduHeading
  {\parbox[t]{0.7\textwidth}{Massachusetts Institute of Technology {\&}\\ Woods Hole Oceanographic Institution}}{Cambridge, MA}
  {Ph.D. in Physical Oceanography}{2021--Present}
\eduHeading
  {University of California, Irvine}{Irvine, CA}
  {B.S. in Mathematics, Concentration in Data Science}{2018--2021}

% -------------------- PUBLICATIONS --------------------
\section{Publications}

\pubListStart
\justifying
\pubItem{\textbf{Meza, A.}, \& Gebbie, G. (2025). Wind-driven mid-depth Pacific cooling in a dynamically consistent ocean state estimate. \textit{Journal of Geophysical Research: Oceans}. doi.org/10.1029/2025JC022462}
% \item \textbf{Meza, A.}, Drake, H. \& Gebbie, G. \textit{(In Prep)}. Projected Changes in Dense Shelf Water and their Abyssal Expression.
% \item \textbf{Meza, A.}, et al. \textit{(In Prep)}. 
% Detection of .
% \item \textbf{Meza, A.}, Bhuyan, P., Zheng, Z. \& Gebbie, G. \textit{(In Prep)}. Passive and Dynamic Contributions to Mean Ocean Temperature during the Last Glacial Maximum.
\pubListEnd

% -------------------- RESEARCH EXPERIENCE --------------------
\section{Research Experience}

\positionHeading[Graduate Research Assistant]
  {Woods Hole Oceanographic Institution}{Woods Hole, MA}{Sep 2021--Present}
\itemListStart
  \myItem{Tested causes of deep ocean cooling using global MITgcm simulations, supporting NASA efforts of global ocean modeling and data assimilation.}
  \myItem{Analyzed 40TB+ of high-resolution coupled climate model output to understand the connections between ocean circulation and dissolved chemicals in the ocean.}
  \myItem{Produced written reports, posters, and presentations to communicate findings to broader communities.}
    \myItem{Processed and analyzed 5TB+ of high-resolution ocean reanalysis data and found significant connections between near-shore sea surface temperature and extreme California precipitation events.}
  \myItem{Developed tools to analyze big climate data using Python and Julia.}
\itemListEnd

\positionHeading[Summer Intern]
  {Los Alamos National Laboratory}{Los Alamos, NM}{Jun 2021--Aug 2021}
\itemListStart
  \myItem{Implemented reduced-precision capabilities within the ocean component of the Energy Exascale Earth System Model.}
  \myItem{Found that reduced precision significantly reduced compute time but at cost of model skill.}
\itemListEnd

\positionHeading[Summer Intern]
  {The Aerospace Corporation}{Los Angeles, CA}{Jun 2020--Sep 2020}
\itemListStart
    \myItem{Designed and implemented reinforcement learning algorithms for adaptive packet routing in satellite network simulations.}
  \myItem{Empirical models were built in Python primarily using PyTorch and NetworkX.}
\itemListEnd


% -------------------- PROJECTS --------------------
\section{Personal Projects}
\ProjectHeading{xbuoy}
\itemListStart
  \myItem{Developed a Python workflow to query the National Data Buoy Center and aggregate data into daily, monthly and yearly NetCDFs.}
\itemListEnd


% -------------------- SKILLS --------------------
% \section{Technical Skills}
% \skillListStart
% \justifying
% \item \emph{Languages}: { \fontsize{11pt}{11pt}\selectfont Python}, { \fontsize{11pt}{11pt}\selectfont Julia}, { \fontsize{11pt}{11pt}\selectfont MATLAB}.
% \item \emph{Developer Tools}: Unix, Git, Github, VS Code.
% \item \emph{Libraries}: { \fontsize{11pt}{11pt}\selectfont Xarray}, { \fontsize{11pt}{11pt}\selectfont Pandas}, { \fontsize{11pt}{11pt}\selectfont PyTorch}, { \fontsize{11pt}{11pt}\selectfont SciPy}, { \fontsize{11pt}{11pt}\selectfont NumPy}, { \fontsize{11pt}{11pt}\selectfont Matplotlib}, { \fontsize{11pt}{11pt}\selectfont Scikit-learn}, { \fontsize{11pt}{11pt}\selectfont JAX}, { \fontsize{11pt}{11pt}\selectfont Keras}, { \fontsize{11pt}{11pt}\selectfont Tensorflow}.
% \skillListEnd

\end{document}
