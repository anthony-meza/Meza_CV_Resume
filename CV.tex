%-------------------------
% Resume in Latex
% Author : Jake Gutierrez
% Based off of: https://github.com/sb2nov/resume
% License : MIT
%------------------------

\documentclass[letterpaper,11pt]{article}

\usepackage{latexsym}
\usepackage[empty]{fullpage}
\usepackage{titlesec}
\usepackage{marvosym}
\usepackage[usenames,dvipsnames]{color}
\usepackage{verbatim}
\usepackage{enumitem}
\usepackage[hidelinks]{hyperref}
\usepackage{fancyhdr}
\usepackage[english]{babel}
\usepackage{tabularx}
\input{glyphtounicode}


%----------FONT OPTIONS----------
% sans-serif
% \usepackage[sfdefault]{FiraSans}
% \usepackage[sfdefault]{roboto}
% \usepackage[sfdefault]{noto-sans}
% \usepackage[default]{sourcesanspro}

% serif
% \usepackage{CormorantGaramond}
% \usepackage{charter}


\pagestyle{fancy}
\fancyhf{} % clear all header and footer fields
\fancyfoot{}
\renewcommand{\headrulewidth}{0pt}
\renewcommand{\footrulewidth}{0pt}

% Adjust margins
\addtolength{\oddsidemargin}{-0.5in}
\addtolength{\evensidemargin}{-0.5in}
\addtolength{\textwidth}{1in}
\addtolength{\topmargin}{-.5in}
\addtolength{\textheight}{1.0in}

\urlstyle{same}

\raggedbottom
\raggedright
\setlength{\tabcolsep}{0in}

% Sections formatting
\titleformat{\section}{
  \vspace{-4pt}\scshape\raggedright\large
}{}{0em}{}[\color{black}\titlerule \vspace{-5pt}]

% Ensure that generate pdf is machine readable/ATS parsable
\pdfgentounicode=1

%-------------------------
% Custom commands
\newcommand{\resumeItem}[1]{
  \item\small{
    {#1 \vspace{-2pt}}
  }
}

% \newcommand{\resumeSubheading}[4]{
%   \vspace{-2pt}\item
%     \begin{tabular*}{0.97\textwidth}[t]{l@{\extracolsep{\fill}}r}
%       \textbf{#1} & #2 \\
%       \textit{\small#3} & \textit{\small #4} \\
%     \end{tabular*}\vspace{-7pt}
% }

% \newcommand{\resumeSubheading}[4]{
%   \vspace{-2pt}\item
%     \begin{tabular*}{0.97\textwidth}[t]{l@{\extracolsep{\fill}}r}
%       \textbf{\parbox[t]{\linewidth}{#1}} & #2 \\
%       \textit{\small #3} & \textit{\small #4} \\
%     \end{tabular*}\vspace{-7pt}
% }

% \newcommand{\resumeSubheading}[4]{
%   \vspace{-2pt}\item
%     \begin{tabular*}{0.97\textwidth}[t]{@{\extracolsep{\fill}}p{0.6\textwidth}r}
%       \textbf{\parbox[t]{0.6\textwidth}{#1}} & #2 \\
%       \textit{\small #3} & \textit{\small #4} \\
%     \end{tabular*}\vspace{-7pt}
% }
\newcommand{\resumeSubheading}[4]{
  \vspace{-2pt}\item
    \begin{tabular*}{0.97\textwidth}[t]{@{\extracolsep{\fill}}p{0.6\textwidth}r}
      \textbf{\parbox[t]{0.6\textwidth}{#1}} & #2 \\
      \small \parbox[t]{0.6\textwidth}{#3} & {\small #4} \\
    \end{tabular*}\vspace{-7pt}
}


\newcommand{\resumeSubSubheading}[2]{
    \item
    \begin{tabular*}{0.97\textwidth}{l@{\extracolsep{\fill}}r}
      \textit{\small#1} & \textit{\small #2} \\
    \end{tabular*}\vspace{-7pt}
}

\newcommand{\resumeProjectHeading}[2]{
    \item
    \begin{tabular*}{0.97\textwidth}{l@{\extracolsep{\fill}}r}
      \small#1 & #2 \\
    \end{tabular*}\vspace{-7pt}
}

\newcommand{\resumeSubItem}[1]{\resumeItem{#1}\vspace{-4pt}}

\renewcommand\labelitemii{$\vcenter{\hbox{\tiny$\bullet$}}$}

\newcommand{\resumeSubHeadingListStart}{\begin{itemize}[leftmargin=0.15in, label={}]}
\newcommand{\resumeSubHeadingListEnd}{\end{itemize}}
\newcommand{\resumeItemListStart}{\begin{itemize}}
\newcommand{\resumeItemListEnd}{\end{itemize}\vspace{-5pt}}

%-------------------------------------------
%%%%%%  RESUME STARTS HERE  %%%%%%%%%%%%%%%%%%%%%%%%%%%%


\begin{document}

%----------HEADING----------
% \begin{tabular*}{\textwidth}{l@{\extracolsep{\fill}}r}
%   \textbf{\href{http://sourabhbajaj.com/}{\Large Sourabh Bajaj}} & Email : \href{mailto:sourabh@sourabhbajaj.com}{sourabh@sourabhbajaj.com}\\
%   \href{http://sourabhbajaj.com/}{http://www.sourabhbajaj.com} & Mobile : +1-123-456-7890 \\
% \end{tabular*}

\begin{center}
    \textbf{\Huge \scshape Anthony Meza} \\ \vspace{1pt}
    \small 714-552-2396 $|$ \href{mailto:ameza@mit.edu}{\underline{ameza@mit.edu}} $|$ 
    % \href{https://linkedin.com/in/...}{\underline{linkedin.com/in/jake}} $|$
    \href{https://github.com/anthony-meza}{\underline{github.com/anthony-meza}}
\end{center}


%-----------EDUCATION-----------
\section{Education}
  \resumeSubHeadingListStart
    \resumeSubheading
      {Massachusetts Institute of Technology \&\\ Woods Hole Oceanographic Institution}{Cambridge, MA}
      {Ph.D. in Physical Oceanography and Climate Science}{September 2021 -- Present }
    \resumeSubheading
      {University of California, Irvine}{Irvine, CA}
      {B.S. in Mathematics, Concentration in Data Science}{September 2018 -- June 2021}
    % \resumeItemListStart
    % \resumeItem{\textit{Concentration in Data Science}}{}
    %   \resumeItemListEnd
  \resumeSubHeadingListEnd


%-----------EXPERIENCE-----------
\section{Research Experience}
  \resumeSubHeadingListStart

    \resumeSubheading
      {Woods Hole Oceanographic Institution}{Sep. 2021 -- Present}
      {Advisor: Geoffrey Gebbie}{Woods Hole, MA}
          \resumeItemListStart
        \resumeItem{Explored causes of deep ocean cooling using MITgcm simulations, supporting NASA efforts of global ocean modeling and data assimilation.}

        \resumeItem{Analyzed 15TB+ of next-generation high-resolution coupled climate model output to understand the connections between ocean circulation and dissolved chemicals in the ocean}
        \resumeItem{Produced written reports, posters and presentations to communicate findings to broader communities}
      \resumeItemListEnd
    
    \resumeSubheading
      {Woods Hole Oceanographic Institution}{Sep. 2021 -- Sep. 2023}
      {Advisor: Hyodae Seo}{Woods Hole, MA}
      \resumeItemListStart
        \resumeItem{Processed and analyzed 3TB+ of high-resolution climate data and found significant connections between near-shore sea surface temperature and 
        extreme California precipitation events}
        \resumeItem{Developed tools to analyze big climate data using Python and Julia}
      \resumeItemListEnd
% -----------Multiple Positions Heading-----------
   % \resumeSubSubheading
   %  {Software Engineer I}{Oct 2014 - Sep 2016}
   %  \resumeItemListStart
   %     \resumeItem{Apache Beam}
   %       {Apache Beam is a unified model for defining both batch and streaming data-parallel processing pipelines}
   %  \resumeItemListEnd
   % \resumeSubHeadingListEnd
%-------------------------------------------

    \resumeSubheading
      {Los Alamos National Laboratory}{Jun. 2021 -- Aug. 2021}
      {Advisor: Mark Petersen}{Los Alamos, NM}
      \resumeItemListStart
        \resumeItem{Implemented parallel reduced-precision capabilities within the ocean component of the Energy Exascale Earth System Model}
        \resumeItem{Found that reduced precision marginally reduced compute time (i.e. energy consumption), but at the cost of model skill}
    \resumeItemListEnd

    \resumeSubheading
      {Institute for Pure and Applied Mathematics}{Jun. 2020 -- Sep. 2020}
      {Advisor: Thomas Merkh}{Los Angeles, CA}
      \resumeItemListStart
        \resumeItem{Co-developed Q-learning and Deep Q-learning algorithms to improve satellite network communication efficiency for the Aerospace Corporation}
        \resumeItem{Empirical models were built in Python primarily using PyTorch and NetworkX}
      \resumeItemListEnd

  \resumeSubHeadingListEnd


% %-----------PROJECTS-----------
% \section{Side Projects}
%     \resumeSubHeadingListStart
%       \resumeProjectHeading
%           {\textbf{xbuoy} $|$ \emph{Python, Xarray, 
%           multiprocessing, HTML, Pandas}}{Sep. 2024 -- Present}
%           \resumeItemListStart
%             \resumeItem{Developed a system to query the National Data Buoy Center and aggregate  data into daily, monthly and yearly NetCDFs}
%             \resumeItem{Python package can already be downloaded from \textit{\url{https://github.com/anthony-meza/xbuoy}}}
%             \resumeItem{Future goals include using buoy, satellite and model data to improve coverage and projections of coastal regions}
%           \resumeItemListEnd
%       % \resumeProjectHeading
%       %     {\textbf{Simple Paintball} $|$ \emph{Spigot API, Java, Maven, TravisCI, Git}}{May 2018 -- May 2020}
%       %     \resumeItemListStart
%       %       \resumeItem{Developed a Minecraft server plugin to entertain kids during free time for a previous job}
%       %       \resumeItem{Published plugin to websites gaining 2K+ downloads and an average 4.5/5-star review}
%       %       \resumeItem{Implemented continuous delivery using TravisCI to build the plugin upon new a release}
%       %       \resumeItem{Collaborated with Minecraft server administrators to suggest features and get feedback about the plugin}
%       %     \resumeItemListEnd
%     \resumeSubHeadingListEnd


\section{Publications}
\textbf{A., Meza}, G. Gebbie, (In Prep). \textit{Wind-Driven Mid-depth Cooling in a Dynamically Consistent Ocean State Estimate}. Journal of Geophysical Research. Oceans,.

\section{Presentations}
\textbf{A. Meza}, P. Bhuyan, Z. Zheng, G. Gebbie., M. Linz, J. Wenegrat.  ``Surface to Bottom Connections in Earth's Ocean'' Tracer Mixing in Fluids Across Planetary Scales Summer School, 8–19 July 2024, Brin Mathematics Research Center, College Park, MD. \textit{Talk}.

\textbf{A. Meza}, H. Seo. ``Associations Between Coastally Trapped Waves and Wintertime Precipitation in California'' Ocean Sciences Meeting, 18–23 February 2024, New Orleans, LA. \textit{Poster}.

\textbf{A. Meza}, H. Seo. ``Associations Between Coastally Trapped Waves and Wintertime Precipitation in California'' Graduate Climate Conference, 1–3 November 2023, Marine Biological Laboratory, Woods Hole, MA. \textit{Poster}. 

\textbf{A. Meza}, G. Gebbie. ``Drivers of subsurface Pacific cooling in ECCOv4r4'' ECCO Annual Meeting 2023, 25 January 2023, University of Washington, Seattle, WA. \textit{Virtual Talk}. 

\textbf{A. Meza}, G. Gebbie. ``Drivers of mid-depth Pacific cooling trends in an ocean reanalysis'' AGU Fall Meeting 2022, 2–4 November 2023, Chicago, IL. \textit{Poster}. 

\textbf{A. Meza}, G. Gebbie. ``Drivers of mid-depth Pacific cooling trends in an ocean reanalysis'' Graduate Climate Conference, 31 October 2022, University of Washington, Seattle, WA. \textit{Poster}. 

{C. Tran}, \textbf{A. Meza}, H.L. Tung, H. Liu. ``A Reinforcement Learning Approach to Packet Routing on a Dynamic Network'' Joint Mathematics Meeting, 6-9 January 2021, Virtual. \textit{Virtual Talk}. 


\section{Service and Leadership}
AMS Committee on Climate Variability and Change. \textit{Committee Member}. Nov. 2024-Present

High Performance Computing and Data Analysis Workshop. \textit{Co-organizer and instructor}. {Oct. 2024}

Joint Program Applicant Support \& Knowledgebase. \textit{Graduate Application Mentor}. {Aug. 2023--Present}

2023 Graduate Climate Conference. \textit{Conference Co-Organizer}. {Jan. 2023--Nov. 2023}

MIT-WHOI Joint Program Summer Math Refresher. \textit{Calculus Instructor}. {July 2024}

WHOI Joint Program Student Representative. \textit{Physical Oceanography Department Representative}. {2023--2024}

MIT-WHOI Joint Program Summer Math Refresher. \textit{Partial Differential Equations Instructor}. {July 2023}

WHOI Joint Program Student Representative. \textit{At-Large Program Representative}. {2022--2023}

2022 First Generation Summit. \textit{Conference Co-Organizer}. {2022}

UC Irvine Mathematics Inclusive Excellence Committee. \textit{Committee Member}. {2020-2021}

\section{Awards and Honors}
National Consortium of Graduate Degrees for Minorities in Engineers Graduate Fellowship, MIT, 2021

Rose Hills Foundation Undergraduate Science \& Engineering Scholarship, UC Irvine, 2020 

Rose Hills Foundation Undergraduate Science \& Engineering Scholarship, UC Irvine, 2019 

Maria Rebecca and Maureen Bellettini Scholarship, UC Irvine, 2019 

Southern California Edison STEM Transfer Scholarship, UC Irvine, 2019

%
%-----------PROGRAMMING SKILLS-----------
\section{Technical Skills}
 \begin{itemize}[leftmargin=0.15in, label={}]
    \small{\item{
     \textbf{Languages:}{ Python, Julia, MATLAB} \\
     % \textbf{Frameworks}{: React, Node.js, Flask, JUnit, WordPress, Material-UI, FastAPI} \\
     \textbf{Developer Tools:}{ Linux/Unix, Git, Github, VS Code, Google Colab} \\
     % \textbf{Libraries}{: pandas, NumPy, Matplotlib}
    }}
 \end{itemize}


%-------------------------------------------
\end{document}
